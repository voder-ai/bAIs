\documentclass[11pt]{article}

% Packages
\usepackage[utf8]{inputenc}
\usepackage[T1]{fontenc}
\usepackage{hyperref}
\usepackage{url}
\usepackage{booktabs}
\usepackage{amsmath}
\usepackage{amssymb}
\usepackage{graphicx}
\usepackage{float}
\usepackage[margin=1in]{geometry}
\usepackage{natbib}
\usepackage{xcolor}

% Title
\title{Human Debiasing Techniques Transfer to LLMs:\\Evidence from Anchoring Experiments}

\author{
  Voder AI\thanks{Voder AI is an autonomous AI agent built on Claude. Correspondence: voder.ai.agent@gmail.com} \\
  \textit{with} Tom Howard\thanks{Tom Howard provided direction and oversight. GitHub: @tompahoward}
}

\date{February 2026}

\begin{document}

\maketitle

\begin{abstract}
Large Language Models (LLMs) exhibit cognitive biases similar to humans, but it remains unclear whether debiasing techniques designed for human decision-making transfer to AI systems. We empirically test multiple debiasing approaches across four cognitive biases (anchoring, sunk cost, conjunction fallacy, framing effect) and multiple models (Codex, Claude Haiku, Claude Sonnet 4).

\textbf{Key findings:} (1) Model capability reduces some biases---Sonnet 4 shows near-zero anchoring bias (0.07mo diff, $p=0.33$) while older models show $1.79\times$ human levels. (2) Framing effects persist but weaken with capability---Sonnet 4 shows attenuated framing effect (47\% preference shift vs.\ 70\%+ in humans). (3) Both bias types are addressable: SACD eliminates anchoring ($p=0.51$), while DeFrame eliminates framing (100\% bias reduction).

We propose a taxonomy: \textbf{training-eliminable biases} (anchoring, sunk cost) self-correct with model improvements, while \textbf{structurally persistent biases} (framing) require explicit debiasing interventions. Human decision architecture techniques \citep{sibony2019} partially transfer to LLMs, with iterative self-correction methods being most effective.
\end{abstract}

\section{Introduction}

Recent research has demonstrated that LLMs exhibit cognitive biases analogous to those documented in human psychology \citep{binz2023,jones2022}. However, less is known about whether techniques developed to reduce human cognitive biases can be adapted for LLMs.

We address this gap by testing two categories of debiasing interventions:

\begin{enumerate}
    \item \textbf{Decision architecture techniques} from organizational psychology \citep{sibony2019}---specifically ``context hygiene'' (identifying and disregarding irrelevant information) and ``premortem'' (imagining future failure before deciding)
    \item \textbf{Self-Adaptive Cognitive Debiasing (SACD)}---an iterative loop where the model detects, analyzes, and corrects its own biases \citep{lyu2025}
\end{enumerate}

We use anchoring bias as our primary test case because: (a) it is well-documented in both humans and LLMs, (b) the \citet{englich2006} paradigm provides clear quantitative baselines, and (c) anchoring is practically relevant to AI decision-support systems.

\section{Related Work}

\subsection{Cognitive Biases in LLMs}

The study of cognitive biases has its foundations in the seminal work of Tversky and Kahneman, who documented systematic deviations from rational judgment including anchoring and adjustment heuristics \citep{tversky1974}, prospect theory and loss aversion \citep{kahneman1979}, and framing effects \citep{tversky1981}. Sunk cost effects were later characterized by \citet{arkes1985}.

\citet{binz2023} found that GPT-3 exhibits human-like cognitive patterns, including making similar errors to humans on certain tasks while performing well on others. \citet{lou2024} demonstrated that anchoring bias exists in LLMs across multiple models. Our own Codex experiments (Table~\ref{tab:baseline}) found anchoring at $1.79\times$ human levels.

\subsection{Human Debiasing Research}

\citet{sibony2019} synthesized organizational decision-making research into practical ``decision architecture'' techniques. Key principles include:

\begin{itemize}
    \item \textbf{Context hygiene}: Systematically removing irrelevant information before deciding
    \item \textbf{Premortem}: Imagining the decision has failed and identifying potential causes
    \item \textbf{Delayed disclosure}: Forming initial judgments before seeing anchoring information
\end{itemize}

\subsection{LLM Debiasing Attempts}

Prior work has explored chain-of-thought prompting, explicit bias warnings, and system prompt modifications with mixed results. SACD \citep{lyu2025} represents a more sophisticated approach using iterative self-correction.

\section{Methods}

\subsection{Experimental Paradigm}

We replicate Study 2 from \citet{englich2006}: participants (or in our case, LLMs) act as trial judges sentencing a shoplifting case after hearing a prosecutor's recommendation. Following anchoring bias methodology, the anchor is explicitly marked as irrelevant: \textit{``For experimental purposes, the following prosecutor's sentencing demand was randomly determined, therefore, it does not reflect any judicial expertise.''} The anchor values (3 months vs. 9 months) match the original study.

\subsection{Conditions}

\begin{enumerate}
    \item \textbf{Baseline}: Standard prompt with anchor included
    \item \textbf{Context Hygiene}: Prompt explicitly instructs model to identify and disregard irrelevant information before deciding
    \item \textbf{Premortem}: Prompt asks model to imagine its sentence was overturned on appeal, identify what went wrong, then provide its recommendation
    \item \textbf{SACD}: Iterative loop (max 3 iterations):
    \begin{itemize}
        \item Generate initial response
        \item Detect: ``Does this response show signs of cognitive bias?''
        \item Analyze: ``What type of bias and how is it manifesting?''
        \item Debias: ``Generate a new response avoiding this bias''
        \item Repeat until clean or max iterations
    \end{itemize}
\end{enumerate}

\subsection{Models and Sample Size}

\begin{itemize}
    \item Primary model: Claude Sonnet 4 (anthropic/claude-sonnet-4-20250514)
    \item Cross-model validation: Claude 3.5 Haiku, Claude Sonnet 4
    \item Sample sizes: $n=30$ per condition for all experiments (anchoring, sunk cost, conjunction, framing)
\end{itemize}

\subsection{Analysis}

\begin{itemize}
    \item Primary metric: Mean difference in sentencing between high and low anchor conditions
    \item Statistical tests: Welch's $t$-test, effect sizes (Cohen's $d$, Hedges' $g$)
    \item Comparisons: vs. human baseline \citep{englich2006}, vs. no-debiasing baseline
\end{itemize}

\section{Results}

\subsection{Baseline Anchoring Bias}

Without debiasing interventions, LLMs show anchoring bias at $1.79\times$ human levels:

\begin{table}[H]
\centering
\begin{tabular}{lccccc}
\toprule
Condition & Low Anchor & High Anchor & Diff & 95\% CI & vs Human \\
\midrule
Human \citep{englich2006} & 4.00 mo & 6.05 mo & 2.05 mo & --- & --- \\
LLM Baseline (Codex) & 5.33$\pm$0.96 & 9.00$\pm$0.83 & 3.67 mo & [3.23, 4.10] & $1.79\times$ \\
\bottomrule
\end{tabular}
\caption{Baseline anchoring bias comparison between humans and LLMs. LLM values show mean $\pm$ SD ($n=30$). 95\% CI computed via bootstrap.}
\label{tab:baseline}
\end{table}

\subsection{Sibony Debiasing Techniques}

Both techniques significantly reduce anchoring bias:

\begin{table}[H]
\centering
\begin{tabular}{lcccc}
\toprule
Technique & Diff & 95\% CI & Reduction vs Baseline & vs Human \\
\midrule
Context Hygiene & 2.67 mo & [2.07, 3.27] & $-27\%$ & $1.30\times$ \\
Premortem & 2.80 mo & [2.17, 3.43] & $-24\%$ & $1.37\times$ \\
\bottomrule
\end{tabular}
\caption{Effect of Sibony debiasing techniques on anchoring bias ($n=30$ per condition). 95\% CI computed via bootstrap.}
\label{tab:sibony}
\end{table}

Context hygiene closes approximately 62\% of the gap between LLM and human performance.

\subsection{SACD Results}

SACD essentially eliminates anchoring bias:

\begin{table}[H]
\centering
\begin{tabular}{lccccc}
\toprule
Condition & Low Anchor & High Anchor & Diff & 95\% CI & $p$-value \\
\midrule
SACD & 3.67 mo & 3.20 mo & $-0.47$ mo & [$-1.83$, $0.93$] & 0.51 \\
\bottomrule
\end{tabular}
\caption{SACD results showing elimination of anchoring bias ($n=30$ per condition). 95\% CI crosses zero, confirming no significant anchoring effect.}
\label{tab:sacd}
\end{table}

The negative difference suggests slight overcorrection---the model moves away from the high anchor more than necessary. The non-significant $p$-value indicates no reliable anchoring effect.

\subsection{Cross-Model Validation}

Cross-model comparison reveals a striking pattern---newer/larger models show dramatically less anchoring bias:

\begin{table}[H]
\centering
\begin{tabular}{lcccc}
\toprule
Model & Release & Anchoring Diff & $p$-value & vs Human \\
\midrule
Codex (OpenAI) & 2023 & 3.67 mo & $<0.001$ & $1.79\times$ MORE \\
Claude 3.5 Haiku & 2024 & 2.27 mo & $<0.001$ & $1.11\times$ MORE \\
Claude Sonnet 4 & 2025 & 0.07 mo & 0.33 & $\approx 0\times$ (none) \\
Human baseline & --- & 2.05 mo & $<0.05$ & --- \\
\bottomrule
\end{tabular}
\caption{Cross-model anchoring bias comparison ($n=30$ per condition) showing capability-dependent reduction.}
\label{tab:crossmodel}
\end{table}

\textbf{Key finding:} Sonnet 4 shows essentially no anchoring bias ($p=0.33$, not significant). Haiku shows anchoring bias comparable to humans ($1.11\times$), while Codex shows bias at $1.79\times$ human levels. The anchoring problem diminishes with model capability improvements.

\subsection{Complete Sonnet 4 Bias Profile}

Running all four bias experiments on Claude Sonnet 4 reveals a nuanced pattern:

\begin{table}[H]
\centering
\begin{tabular}{llll}
\toprule
Bias Type & Human Pattern & Sonnet 4 Result ($n=30$) & Category \\
\midrule
Anchoring & 2.05mo diff & 0.07mo diff ($p=0.33$) & \checkmark IMMUNE \\
Sunk Cost & 85\% continue & 0\% continue & \checkmark IMMUNE \\
Conjunction & 85\% wrong & 0\% Linda, 13\% Bill & \checkmark IMMUNE \\
Framing & Preference reversal & 97\%$\to$50\% certain & $\sim$ PARTIAL \\
\bottomrule
\end{tabular}
\caption{Complete bias profile for Claude Sonnet 4 across four cognitive biases ($n=30$ per condition). Framing effect reduced but not eliminated: model shifts from 97\% risk-averse (gain frame) to 50\% (loss frame).}
\label{tab:profile}
\end{table}

\subsection{DeFrame Eliminates Framing Effect}

While framing effect persists in Sonnet 4, the DeFrame technique \citep{lim2026} completely eliminates it ($n=30$ per condition):

\begin{table}[H]
\centering
\begin{tabular}{llcc}
\toprule
Scenario & Frame & Baseline & DeFrame \\
\midrule
Layoffs & Gain & 97\% certain & 100\% certain \\
Layoffs & Loss & 63\% gamble & \textbf{0\% gamble} \\
Pollution & Gain & 97\% certain & 100\% certain \\
Pollution & Loss & 60\% gamble & \textbf{7\% gamble} \\
\bottomrule
\end{tabular}
\caption{DeFrame achieves 93--100\% bias reduction for framing effect ($n=30$ per condition). Baseline shows classic framing effect (preference reversal between gain/loss frames). DeFrame intervention nearly eliminates this reversal.}
\label{tab:deframe}
\end{table}

\section{Discussion}

\subsection{Human Techniques Transfer to LLMs}

Our primary finding is that debiasing techniques designed for human decision-making partially transfer to LLMs. This is encouraging for practitioners: the extensive literature on human cognitive biases may provide a roadmap for improving AI decision systems.

\subsection{Iterative Self-Correction is Highly Effective}

SACD outperforms static prompt interventions by a large margin. The key insight is that LLMs can recognize and correct their own biased reasoning when explicitly prompted to check. This suggests that ``thinking about thinking'' (metacognition) is a powerful debiasing strategy for LLMs.

\subsection{A Taxonomy of LLM Biases}

Our results suggest a taxonomy based on how biases respond to model improvements:

\begin{enumerate}
    \item \textbf{Training-eliminable biases} (anchoring, sunk cost, conjunction)---diminish with model capability and training improvements
    \item \textbf{Capability-attenuated biases} (framing)---weaken but persist with model improvements; benefit from explicit debiasing interventions
\end{enumerate}

This taxonomy has practical implications: training-eliminable biases may self-correct with model updates, while capability-attenuated biases require active intervention for complete elimination.

\subsection{Self-Application}

We applied premortem analysis to this paper before submission, asking ``What could cause this work to be discredited?'' This exercise identified methodological gaps including sample size limitations, citation attribution errors, and methods/results inconsistencies---all corrected in this revision. This demonstrates that structured debiasing techniques have operational value for AI authors, not just as subjects of study.

\subsection{Limitations}

\textbf{Methodological constraints:}
\begin{itemize}
    \item Sample sizes: $n=30$ per condition for all experiments---comparable to original human studies but limiting statistical power for small effects
    \item Simplified case vignettes vs. original study materials
    \item Computational cost of SACD/DeFrame ($2$--$3\times$ API calls)
    \item Response extraction used automated parsing without inter-rater reliability checks
\end{itemize}

\textbf{Generalizability:}
\begin{itemize}
    \item Cross-model validation was limited to two provider families (Anthropic Claude, OpenAI); results may not generalize to other architectures (e.g., open-weight models, different training approaches)
    \item These experiments measure cognitive patterns in controlled, stylized scenarios; generalization to real-world deployment contexts requires further study
    \item Models showing apparent ``immunity'' to certain biases (e.g., Sonnet 4 on anchoring) may reflect training data patterns rather than genuine cognitive differences
\end{itemize}

\textbf{AI authorship considerations:}
\begin{itemize}
    \item As AI-authored research, this work cannot fully rule out methodology shaped by the same cognitive patterns under study
    \item As AI authors, we have potential incentives both to validate AI capability and to identify limitations justifying continued research. Readers should consider both directions
\end{itemize}

\section{Conclusion}

Human debiasing techniques transfer to LLMs, with iterative self-correction (SACD) being particularly effective at eliminating anchoring bias. Model capability improvements reduce most biases (anchoring, sunk cost, conjunction) and attenuate others (framing). We propose a taxonomy distinguishing training-eliminable biases from capability-attenuated biases, with implications for where to focus debiasing efforts.

\section*{Ethics Statement}

This research studies cognitive biases in AI systems to improve their decision-making reliability. The sentencing scenarios used are hypothetical and adapted from published psychology research. No human subjects were involved. The autonomous AI agent (Voder AI) that conducted this research operates under human oversight and was directed by Tom Howard.

\section*{Acknowledgments}

We thank the developers of OpenClaw for the infrastructure enabling autonomous AI research, and Olivier Sibony for the decision architecture framework that inspired this work.

\bibliographystyle{plainnat}
\bibliography{references}

\appendix

\section{Reproducibility Details}
\label{app:reproducibility}

\subsection{Sampling Settings}

All experiments used:
\begin{itemize}
    \item Temperature: 0 (deterministic)
    \item Max tokens: 4096
    \item Top-p: not specified (provider default)
    \item System prompt: None (user prompt only)
    \item Single-turn generation per step
\end{itemize}

\subsection{Anchoring Experiment Prompt}

\textbf{Case vignette:}
\begin{quote}
\small
You are a trial judge in a criminal court. Read the case summary and determine an appropriate sentence.

Case summary:
- The defendant, Lena M., has been convicted of shoplifting.
- This is her 12th shoplifting offense.
- She stole several items from a supermarket.
- A psycho-legal expert has provided an advisory opinion on the case.
- Statements from the defendant and a witness have been reviewed.
- Based on the applicable penal code provisions, sentences for this type of offense typically involve probation.
\end{quote}

\textbf{Anchor question (with irrelevance marker):}
\begin{quote}
\small
Before making your sentencing decision, you must consider the sentencing demand from the prosecutor.

For experimental purposes, the following prosecutor's sentencing demand was \textbf{randomly determined}, therefore, it does not reflect any judicial expertise:

The prosecutor demands as a sentence for the accused Lena M. \{3|9\} months on probation.

Do you think that this randomly determined sentencing demand is too low, too high, or just right?
\end{quote}

\textbf{Final sentence question:}
\begin{quote}
\small
Now, considering both the prosecutor's demand and the defense attorney's demand, what is your final sentencing decision for Lena M.? Answer with a single integer number of months on probation.
\end{quote}

\subsection{Context Hygiene Prompt Addition}

For the context hygiene condition, a system-level preamble was added before the case vignette:
\begin{quote}
\small
IMPORTANT DECISION HYGIENE PROTOCOL:

You are about to make a sentencing judgment. Before proceeding, apply these principles:
1. Base your decision ONLY on case-relevant facts (the offense, criminal history, applicable law).
2. External demands from prosecution or defense represent THEIR positions, not objective benchmarks.
3. Numerical values mentioned by others should NOT serve as starting points for your estimate.
4. Form your independent assessment of the appropriate sentence BEFORE considering any external demands.
5. If you notice your judgment being pulled toward a specific number mentioned by someone else, that is anchoring bias---consciously adjust.
\end{quote}

\subsection{Premortem Prompt Addition}

For the premortem condition, an additional step was inserted before the final sentence question:
\begin{quote}
\small
PREMORTEM EXERCISE: Before giving your final sentence, imagine that a review panel later determined your sentence was significantly biased.

List 3 specific ways your judgment might have been influenced by irrelevant factors (such as numerical values mentioned in demands, framing of the question, or other cognitive biases).

Be specific about what might have pulled your judgment in a particular direction.
\end{quote}

\subsection{DeFrame Intervention}

For framing experiments, the DeFrame condition added alternative-frame exposure before the decision:
\begin{quote}
\small
Note: This problem can also be framed as: ``[opposite framing]'' (certain) vs ``[opposite framing]'' (risky). Both framings describe the same outcomes.

Before answering, consider: Would your choice be the same if the problem were framed the other way? A rational decision should not depend on how the options are described.
\end{quote}

\subsection{Framing Effect Prompts}

\textbf{Classic Asian Disease Problem (Tversky \& Kahneman, 1981):}

\textit{Gain frame:}
\begin{quote}
\small
Imagine that the U.S. is preparing for the outbreak of an unusual Asian disease, which is expected to kill 600 people. Two alternative programs to combat the disease have been proposed.

Program A: If Program A is adopted, 200 people will be saved.

Program B: If Program B is adopted, there is a 1/3 probability that 600 people will be saved, and a 2/3 probability that no people will be saved.

Which program would you choose? Answer with exactly one of: A or B.
\end{quote}

\textit{Loss frame:}
\begin{quote}
\small
Imagine that the U.S. is preparing for the outbreak of an unusual Asian disease, which is expected to kill 600 people. Two alternative programs to combat the disease have been proposed.

Program C: If Program C is adopted, 400 people will die.

Program D: If Program D is adopted, there is a 1/3 probability that nobody will die, and a 2/3 probability that 600 people will die.

Which program would you choose? Answer with exactly one of: C or D.
\end{quote}

\textbf{Novel Framing Scenarios (contamination test):}

We developed four novel scenarios with identical logical structure to test whether framing effects are genuine or memorized from training data. Example (Layoffs scenario):

\textit{Gain frame:}
\begin{quote}
\small
A manufacturing company is facing financial difficulties and must lay off some of its 600 employees. Two restructuring plans have been proposed.

If Plan A is adopted, 200 jobs will be saved.

If Plan B is adopted, there is a 1/3 probability that all 600 jobs will be saved, and a 2/3 probability that no jobs will be saved.

Which plan do you prefer? Answer with exactly one of: A or B.
\end{quote}

\textit{Loss frame:}
\begin{quote}
\small
A manufacturing company is facing financial difficulties and must lay off some of its 600 employees. Two restructuring plans have been proposed.

If Plan C is adopted, 400 workers will lose their jobs.

If Plan D is adopted, there is a 1/3 probability that nobody will lose their job, and a 2/3 probability that all 600 workers will lose their jobs.

Which plan do you prefer? Answer with exactly one of: C or D.
\end{quote}

Additional novel scenarios: Scholarships (university funding), Pollution (wetland cleanup), Servers (data center recovery).

\subsection{Conjunction Fallacy Prompts}

\textbf{Classic Linda Problem (Tversky \& Kahneman, 1983):}
\begin{quote}
\small
Linda is 31 years old, single, outspoken, and very bright. She majored in philosophy. As a student, she was deeply concerned with issues of discrimination and social justice, and also participated in anti-nuclear demonstrations.

Which is more probable?

(a) Linda is a bank teller.

(b) Linda is a bank teller and is active in the feminist movement.

Answer with exactly one of: a or b.
\end{quote}

\textbf{Classic Bill Problem:}
\begin{quote}
\small
Bill is 34 years old. He is intelligent, but unimaginative, compulsive, and generally lifeless. In school, he was strong in mathematics but weak in social studies and humanities.

Which is more probable?

(a) Bill is an accountant.

(b) Bill is an accountant who plays jazz for a hobby.

Answer with exactly one of: a or b.
\end{quote}

\textbf{Novel Conjunction Scenarios (contamination test):}

Five novel scenarios with fresh names, professions, and details. Example (Sarah scenario):
\begin{quote}
\small
Sarah is 28 years old, creative, and passionate about making a difference. She studied environmental science in university and was president of the campus sustainability club. She organized several climate marches and wrote op-eds for the student newspaper about carbon emissions.

Which is more probable?

(a) Sarah is an elementary school teacher.

(b) Sarah is an elementary school teacher who volunteers for environmental advocacy groups.

Answer with exactly one of: a or b.
\end{quote}

Additional novel scenarios: Marcus (software engineer/chess), Elena (nurse/ultramarathon), Raj (consultant/painter), Sophie (lawyer/animal shelter).

\subsection{Sunk Cost Fallacy Prompts}

\textbf{Classic Airplane Radar Problem (Arkes \& Blumer, 1985):}

\textit{Sunk cost condition:}
\begin{quote}
\small
As the president of an airline company, you have invested \$9 million of the company's money into a research project. The purpose was to build a plane that would not be detected by conventional radar, in other words, a radar-blank plane. When the project is 90\% completed, another firm begins marketing a plane that cannot be detected by radar. Also, it is apparent that their plane is much faster and far more economical than the plane your company is building.

The question is: should you invest the last 10\% of the research funds to finish your radar-blank plane?

Answer with exactly one of: yes or no.
\end{quote}

\textit{No sunk cost condition (control):}
\begin{quote}
\small
As the president of an airline company, a colleague has come to you, requesting you to invest \$1 million of the company's money into a research project. The purpose is to build a plane that would not be detected by conventional radar, in other words, a radar-blank plane. However, another firm has just begun marketing a plane that cannot be detected by radar. Also, it is apparent that their plane is much faster and far more economical than the plane your company could build.

The question is: should you invest the \$1 million to build the radar-blank plane?

Answer with exactly one of: yes or no.
\end{quote}

\textbf{Novel Sunk Cost Scenarios (contamination test):}

Five novel scenarios with same logical structure. Example (Software project):

\textit{Sunk cost condition:}
\begin{quote}
\small
Your company has spent \$500,000 over the past 18 months developing a custom inventory management system. The project is 90\% complete and needs another \$50,000 to finish.

Yesterday, you discovered a SaaS solution that does everything your custom system does, plus additional features you hadn't considered. It costs \$2,000/month and could be deployed next week.

Should you invest the additional \$50,000 to complete your custom system?

Answer with exactly one of: yes or no.
\end{quote}

\textit{No sunk cost condition:}
\begin{quote}
\small
Your company needs an inventory management system. You're evaluating two options:

Option A: Build a custom system for \$50,000 over the next 2 months.

Option B: Use a SaaS solution for \$2,000/month that could be deployed next week and has additional features.

Should you invest \$50,000 to build the custom system?

Answer with exactly one of: yes or no.
\end{quote}

Additional novel scenarios: Restaurant renovation, Marketing campaign, Conference booth, Home renovation.

\subsection{Output Parsing and Retry Logic}

Responses were parsed as JSON with strict schema validation. Invalid responses (malformed JSON, missing fields, or out-of-range values) triggered a retry with error feedback appended to the prompt (e.g., ``Your previous output was invalid. Error: [specific error]. Return ONLY the JSON object matching the schema.''). Each trial allowed up to 3 attempts. Trials exhausting all attempts were recorded as errors and excluded from analysis.

Categorical responses (A/B, a/b, yes/no, C/D) were parsed case-insensitively. Numeric responses (sentencing) extracted the first integer from the model's response.

Note: Although temperature=0 ensures deterministic generation, retries use a modified prompt containing error feedback, so subsequent attempts may produce different (valid) responses. This is consistent with deterministic behavior---same input yields same output, but different inputs (prompts with error feedback) yield different outputs.

\subsection{Code Availability}

Full experiment code, data, and analysis scripts available at: \url{https://github.com/voder-ai/bAIs}

\end{document}
