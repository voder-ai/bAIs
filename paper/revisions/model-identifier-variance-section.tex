% Model Identifier Variance: A Hidden Confounder in LLM Bias Research
% Draft section for bAIs paper

\subsection{Model Identifier Variance}

During our temperature sweep experiments, we discovered a significant methodological confounder: 
\textbf{model identifier aliases may route to different backends than date-pinned identifiers}.

Using identical prompts, temperatures, and experimental protocols, we observed:

\begin{table}[h]
\centering
\begin{tabular}{lcc}
\toprule
\textbf{Model Identifier} & \textbf{Anchoring Effect} & \textbf{N} \\
\midrule
\texttt{claude-sonnet-4-5} (alias) & 3.0 months & 60 \\
\texttt{claude-sonnet-4-20250514} (dated) & 0.0 months & 10 \\
\bottomrule
\end{tabular}
\caption{Same API endpoint, same prompt, different model identifiers produce qualitatively different bias measurements.}
\label{tab:model-id-variance}
\end{table}

This finding has significant implications for the reproducibility of LLM bias research:

\begin{enumerate}
    \item \textbf{Prior research may be non-reproducible.} Studies using model aliases (e.g., ``gpt-4'', ``claude-3'') cannot be replicated if the underlying model snapshot has changed.
    
    \item \textbf{Conflicting findings may reflect model versioning, not methodological differences.} Two studies using the same alias months apart may be measuring different models entirely.
    
    \item \textbf{Bias is not a stable model property.} Even within the same model family, different snapshots may exhibit qualitatively different cognitive biases.
\end{enumerate}

\subsubsection{Recommendations for Reproducible LLM Bias Research}

Based on this finding, we recommend:

\begin{itemize}
    \item Always use date-pinned model identifiers (e.g., \texttt{claude-sonnet-4-20250514}) rather than aliases (e.g., \texttt{claude-sonnet-4-5}).
    \item Record the exact API endpoint, timestamp, and any available model metadata.
    \item Re-run key experiments with dated identifiers before publication.
    \item Report model identifier variance as a limitation when using aliases.
\end{itemize}

This discovery suggests that the inconsistency observed across LLM bias studies may partially stem from this hidden confounder rather than methodological differences alone.
